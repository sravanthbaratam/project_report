%%%%%%%%%%%%%%%%%%%%%%%%%%%%%%%%%
%\chapter{\LARGE Conclusions and Scope for Future Work}
\chapter{Conclusions and Scope for Future Work}
%%%%%%%%%%%%%%%%%%%%%%%%%%%%%%%%%
\section{Conclusions}
%%%%%%%%%%%%%%%%%%%%%%%%%%%%%%%%%%%
In this thesis, an attempt has been made to model input-output 
relationships of electron beam welding process. Here, there 
are many factors, which are responsible to generate good quality 
welds. In the
present study, however,  three inputs are utilized to model the 
process. Various bead-geometric parameters like bead height (not in 
case of Al-1100), bead width
and bead penetration have been considered as the outputs along with the
coordinates of three intuitively  defined points on the weld profile, 
and micro-hardness of the weld-bead.
% Chapter 4
\begin{itemize}

\item As a first step in this study, the experiments have been 
conducted 
as per a CCD matrix. This has been done, so that the entire search 
space is properly covered within the experimental arena. Regression 
equations have been obtained with the experimental and measured data. 
As the experiments are conducted as per a CCD, the expected curvatures 
in the surface plots have been noticed, if they exist. From the surface 
plots of BH,  we conclude that in order to get minimum BH, the welding 
should be carried out at the lower accelerating voltage and beam current 
but at the same time at higher welding speed. Similarly, we conclude 
that in order to get minimum BW, the welding should be conducted at 
lower beam current and higher welding speed. However, in order to get 
the maximum BP, the welding should be performed at a lower speed and 
higher accelerating voltage and beam current.

\item To effectively utilize the energy to weld two pieces together, the  maximum energy should penetrate along the depth and minimum should flow laterally. In other words, the welded area should be kept to its minimum value, while increasing the penetration to its maximum. This has been accomplished using a GA. A GA with a penalty function approach has been utilized to solve this constrained optimization problem.


% Chapter 5
\item In case of electron beam welding, as it is a high density welding, 
the weldment will have a  dagger shape instead of customary 
elliptical. Though a dagger shape is a complex one and is difficult to 
predict, an intuitive approach has been used in the thesis to 
predict the complex shape to a great success. The GA-NN is found to 
perform better than the BPNN in predicting the bead profiles. Being  a 
gradient-based method, the chance of the BPNN algorithm to get trapped 
into the local minima is more. On the other hand, the probability of the 
GA-solutions to get stuck at the local minima is less. Thus, in case of 
predicting  weld profiles of Al-1100, GA-NN is seen to outperform the 
BPNN. However, in case of ASS-304, BPNN is found to perform better than 
the GA-NN in predicting the bead profiles. It has happened so, as the 
performances of GA and BP algorithm are dependent on the nature of error 
functions. For a uni-modal error function (which cannot be predicted 
before-hand), the BP algorithm may outperform the GA. Thus, there is no 
guarantee that GA-NN will always perform better than the BPNN.






% Chapter 6

\item In the field of soft computing, as we have to often deal with a  
huge data set to train networks, the arena of data clustering has been 
ventured in this thesis. The performances of the existing popular fuzzy 
clustering techniques have been compared with that of one developed 
clustering method. It has been shown in the thesis that the newly 
developed algorithm can cluster the data with greater compactness and 
distinctness than the existing fuzzy clustering algorithms do. Both the 
developed algorithm and EFC algorithm have yielded more 
distinct clusters compared to the FCM algorithm. It is also 
interesting to note that for one data set (that is, Al-1100), the 
proposed algorithm has outperformed the other two algorithms in 
terms of distinctness. The clusters obtained by the newly developed  
algorithm and FCM algorithm are seen to be more compact compared to 
those obtained by the EFC algorithm. It is also important to observe 
that almost the same level of compactness has been achieved by both the 
newly developed as well as FCM algorithms for ASS-304 data set.
The performances of clustering algorithms are found to be data-dependent.






% Chapter 7

\item Finally, the input-output relationships of EBW process have been 
modeled in both forward as well as reverse directions using radial basis 
function neural networks. In order to decide the number of 
hidden neurons, input-output data points have been clustered based on the 
similarity among the data points using two well-known fuzzy clustering 
algorithms and one newly proposed algorithm. Three approaches are 
developed for the forward and reverse mappings each by utilizing three 
different clustering techniques. The performances of the developed 
approaches are tested on two different data sets. All these three 
approaches are found to carry out the forward and reverse mappings 
successfully within a reasonable accuracy limit. It has happened so, due 
to the fact that a hybrid scheme of optimization has been adopted in the 
said three approaches, where a global optimizer like GA has been used 
simultaneously along with a local optimizer, namely BP algorithm. The 
difference in the performances of these approaches has come due to the 
application of different clustering algorithms. Approach 3 is found to 
be the best out of all the approaches. It may be due to the supremacy of 
the newly developed clustering technique over the other two existing 
clustering techniques. To automate any process, its input-output relationships are to be known in both the forward and reverse directions, on-line. Thus, the present study may be considered as an important step towards automating a process.
\end{itemize}
%%%%%%%%%%%%%%%%%%%%%%%%%%%%%%%%%
\section{Scope for Future Work}
%%%%%%%%%%%%%%%%%%%%%%%%%%%%%%%%%%%
The present study can be extended in a number of ways, as stated below.
\begin{itemize}
\item This thesis deals with the study on BOP welding carried out on 
stainless steel (ASS-304) and aluminum (Al-1100) plates. Similar study 
can be conducted for a more practical welding process like Butt welding, 
and others.

\item Clustering of input-output data has been done using three 
algorithms only, which can be attempted in future using some other 
algorithms.

\item Forward and reverse modelings of EBW process have been carried out 
utilizing RBFNN. However, some fuzzy logic-based approaches may also be 
tried for the same  in future.

\item On-line condition monitoring of the EBW process may be attempted 
in future.

\item It is understood that the output of this welding depends on the input parameters like accelerating voltage, beam current, welding speed, beam radius, extent of defocussing and beam power distribution. All these six input parameters could not be considered in the present study due to some practical problems in the set-up.  The same may be tried out in future.
\end{itemize}
