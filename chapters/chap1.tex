%%%%%%%%%%%%%%%%%%%%%%%%%%%%%%%%%%%%%%%%%%%%%
%\chapter{\LARGE Introduction}   
\chapter{ Organisation of Thesis/Dissertation/ Report} 
%%%%%%%%%%%%%%%%%%%%%%%%%%%%%%%%%%%%%%%%%%%%%
 
Utmost attention must be paid to the contents of the thesis/ dissertation/ report, which is being submitted in partial fulfillment of the requirements of the respective degree, it is imperative that a standard format be prescribed. Write up of the thesis may be prepared either in M.S. word or \LaTeX. The format of the \textbf{Provisional Thesis} and \textbf{Final Thesis} are same. The format provided shall be equally applicable for all PG and UG courses.
 
 The \LaTeX   format of the thesis for \textbf{NIT Agartala} (refer to Chapters \textbf{\ref{chap:Auxfrmt} \& \ref{chap:introLatex}}) has been prepared, with inspiration from the documentation and guidelines laid down, for the purpose, by IIT Kharagpur and IIT Madras.
%%%%%%%%%%%%%%%%%%%%%%%%%%%%%%%%%%%%%%%%%%%%%%%%%%%%%%%%

%%%%%%%%%%%%%%%%%%%%%%%%%%%%%%%%%%%%%%%%%%%%%%%%%%%%%%%%
\par This thesis shall be presented in a number of chapters, starting with Introduction and ending with Summary and Conclusions. Each of the other chapters will have a precise title reflecting the contents of the chapter. A chapter can be subdivided into sections, \emph{subsections} and \emph{subsubsection} so as to present the content discretely and with due emphasis.
%%%%%%%%%%%%%%%%%%%%%%%%%%%%%%%%%%%%%%%%%%%%
\section{Introduction}
%%%%%%%%%%%%%%%%%%%%%%%%%%%%%%%%%%%%%%%
\par The title of \textbf{Chapter 1} of a thesis shall be Introduction. It shall justify and highlight the problem posed, define the topic and explain the aim and scope of the work presented in the thesis on the basis of the literatures cited. It may also highlight the significant contributions from the investigation. 

\section{Report of Present Investigation}
The reporting on the investigation shall be presented in the next chapters with appropriate chapter title.
\begin{itemize}
	\item 	Due importance shall be given to experimental setups, procedures adopted, techniques developed, methodologies developed and adopted.
	\item  While important derivations/formula should normally be presented in the text of these chapters, extensive and long treatments, copious details and tedious information, detailed results in tabular and graphical forms may be presented in the Appendices. Representative data in table and figures may, however, be included in appropriate chapters.
	\item  Figures and tables should be presented immediately following their first mention in the text. Short tables and figures (say, less than half the writing area of the page) should be presented within the text, while large table and figures may be presented on separate pages.
	\item  Equations should form separate lines with appropriate paragraph separation above and below the equation line, with equation numbers flushed to the right. Equation number of all the Equation should be chapter wise. The number of Equations should follow right alignment.
\end{itemize}
%%%%%%%%%%%%%%%%%%%%%%%%%%
\section{Results and Discussions}
%%%%%%%%%%%%%%%%%%%%%%%%%%
This shall form the penultimate chapter of the thesis and shall include a thorough evaluation of the investigation carried out and bring out the contributions from the study. The discussion shall logically lead to inferences and conclusions as well as scope of possible further future work.
%%%%%%%%%%%%%%%%%%%%%%%%%%%%%%%%%%%%%%%%
\section{Summary and Conclusions}
%%%%%%%%%%%%%%%%%%%%%%%%%%%%%%%%%%%%%%%%%%%%%%
This will be the final chapter of the thesis. A brief report of the work carried out shall form the first part of the Chapter. Conclusions derived from the logical analysis presented in the Results and Discussions Chapter shall be presented and clearly enumerated, each point stated separately. Scope for future work should be stated lucidly in the last part of the chapter.
\section{Appendix}
Detailed information, lengthy derivations, raw experimental observations, etc. are to be presented in the separate appendices, which shall be numbered in Roman Capitals e.g. \textbf{Appendix A}.
\section{Literature Cited} 
The list of references should appear at the end of the thesis marked as \textbf{Bibliography}. The references may be listed either alphabetically or sequentially as they appear in the text of the thesis. If pertinent works have been consulted, but not specifically cited, they should be listed as Bibliography or General References. Spacing and font size should be consistent inside a single reference, and there should be double spaced between two different references. The font size and line spacing of each reference shall follow the format of original write-up. The reference format of different category of references is given next \textbf{chapter}. \LaTeX users are advised to refer to \textbf{Chapter \ref{chap:introLatex}} of this document.
\section{Publications by the candidate}
Articles, technical notes, etc. on the topic of the thesis published by the candidate may be separately listed as \textbf{Publications of the Scholar}, after \textbf{Curriculum Vitae}. This may also be included in the contents.

\par The \LaTeX users are advised to use \verb+publications.bib+,provided in the folder as mentioned in  \textbf{Chapter \ref{chap:introLatex}}. Then, fill up the fields and compile.
%%%%%%%%%%%%%%%%%%%%%%%%%%%

