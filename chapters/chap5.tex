\chapter{Introduction to \LaTeX}
\label{chap:introLatex}

This document provides information about the use of \LaTeX in thesis/dissertation/report writing for NIT Agartala. 

All the required files are bundled into a folder \verb+Thesis-NITA-2018+ . The folder is available at the homepage of the MIS of the institute. The thesis/dissertation/report is generated with help of a \textbf{modified} \verb+report.cls+. The thesis will be generated (double sided pages) by compling the \verb+thesis-NITA.tex+ file .  This document provides several useful tips to use \LaTeX, when you
write your thesis.

Before reading any further please note that you are strongly advised
against changing any of the formatting options used in the class
provided in this directory, unless you are absolutely sure that it
does not violate the NITA formatting guidelines.  \emph{Please do not
  change the margins or the spacing.}  

It is also a good idea to take a quick look at the formatting
guidelines.  Your office or advisor should have a copy.  If they
don't, pester them, they really should have the formatting guidelines
readily available somewhere.

To compile the \LaTeX program windows users may be advised to download and install either \textbf{TexLive} or \textbf{TeXstudio} and \textbf{MikTeX} 


\section{Example Figures and tables}

Fig.~\ref{fig:nita} shows a simple figure for illustration along with
a long caption.  The formatting of the caption text is automatically
single spaced and indented.  Table~\ref{tab:sample} shows a sample
table with the caption placed correctly.  The caption for this should
always be placed before the table as shown in the example.


\begin{figure}[htpb]
  \begin{center}
    \resizebox{50mm}{!} {\includegraphics *{nita.eps}}
    \resizebox{50mm}{!} {\includegraphics *{nita.eps}}
    \caption {Two NITA logos in a row.  This is also an
      illustration of a very long figure caption that wraps around two
      two lines.  Notice that the caption is single-spaced.}
  \label{fig:nita}
  \end{center}
\end{figure}

\begin{table}[htbp]
  \caption{A sample table with a table caption placed
    appropriately. This caption is also very long and is
    single-spaced.  Also notice how the text is aligned.}
  \begin{center}
  \begin{tabular}[c]{|c|r|} \hline
    $x$ & $x^2$ \\ \hline
    1  &  1   \\
    2  &  4  \\
    3  &  9  \\
    4  &  16  \\
    5  &  25  \\
    6  &  36  \\
    7  &  49  \\
    8  &  64  \\ \hline
  \end{tabular}
  \label{tab:sample}
  \end{center}
\end{table}

\section{Bibliography with BIB\TeX}

We recommend that you use BIB\TeX\ to automatically generate
your bibliography.  It makes managing your references much easier.  It
is an excellent way to organize your references and reuse them.  You
can use one set of entries for your references and cite them in your
thesis, papers and reports.  If you haven't used it anytime before
please invest some time learning how to use it.  

A a simple example BIB\TeX\ file along in this directory
called \verb+sample.bib+ is included in the folder.  The \verb+thesis-NITA.tex+ uses \verb+natbib+ in this thesis  to format the references along with a customized bibliography
style called unsrtnat.
\par Documentation for the \verb+natbib+ package should
be available in your distribution of \LaTeX.  Basically, to cite the
author along with the author name and year use \verb+\cite{key}+ where
\verb+key+ is the citation key for your bibliography entry.  You can
also use \verb+\citet{key}+ to get the same effect.  To make the
citation without the author name in the main text but inside the
parenthesis use \verb+\citep{key}+.  The following paragraph shows how
citations can be used in text effectively.


\section{Other useful \LaTeX\ packages}

The following packages might be useful when writing your thesis.

\begin{itemize}  
\item It is very useful to include line numbers in your document.
  That way, it is very easy for people to suggest corrections to your
  text.  It is recommend the use of the \texttt{lineno} package for this
  purpose.  This is not a standard package but can be obtained on the
  internet.  The directory containing this file should contain a
  lineno directory that includes the package along with documentation
  for it.

\item The \texttt{listings} package should be available with your
  distribution of \LaTeX.  This package is very useful when one needs
  to list source code or pseudo-code.

\item For special figure captions the \texttt{ccaption} package may be
  useful.  This is specially useful if one has a figure that spans
  more than two pages and you need to use the same figure number.

\item The notation page can be entered manually or automatically
  generated using the \texttt{nomencl} package.

\end{itemize}

More details on how to use these specific packages are available along
with the documentation of the respective packages.
