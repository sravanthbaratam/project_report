\chapter{Thesis Format}
\section{Paper quality}
The thesis shall be printed/ photocopied on white bond paper, whiteness 95% or above, weight 70 gram or more per square meter.
\section{Paper size}
The size of the paper shall be standard A 4; height 297 mm, width 210 mm.
\section{Type Setting, Text Processing and Printing}
The text shall be printed employing laser jet or Inkjet printer, the text having been processed using a standard text processor. The standard font shall be \textbf{Times New Roman} of \textbf{12 pts} with \textbf{1.5 line} spacing.
\subsection{Left and Right Margins}
The candidates shall have to take double sided printing with mirrored left (38 mm) and right (25 mm) margin.
\subsection{Header})
The header must have the Chapter number and Section number (e.g., Chapter 2, Section 3) on even numbered page headers and Chapter title or Section title on the odd numbered page header.
\subsection{Paragraph}
Vertical space between paragraphs shall be about 2.5 line spacing. The first line of each paragraph should normally be indented by five characters or 12mm. A candidate may, however, choose not to indent if (s)he has provided sufficient paragraph separation.
A paragraph should normally comprise more than one line. A single line of a paragraph shall not be left at the top or bottom of a page (i.e., no windows or orphans should be left).
\section{Chapter and Section format}
\subsection{Chapter}
Each chapter shall begin
\begin{itemize}
	\item On a fresh odd number page with an additional top margin of about 75mm. Chapter number (in Arabic) and title shall be printed at the center of the line in 6mm font size (18pt) in bold face using upper case alphabets).
	\item A vertical gap of about 25mm shall be left between the Chapter number and Chapter title lines and between chapter title line and the first paragraph.
\end{itemize}
 
 \subsection{Sections and Subsections}
 A chapter can be divided into Sections, Subsections and Subsubsections so as to present different concepts separately. Sections and subsections can be numbered using decimal points, e.g. 2.2 for the second section in Chapter 2 and 2.3.4 for the fourth Subsection in the third Section of Chapter 2. \textbf{Chapters}, \textbf{Sections} and \textbf{Subsections} shall be included in the contents with page numbers flushed to the right. Further subsections need not be numbered or included in the contents. 
 \par The Section and Subsection titles along with their numbers in 5 and 4mm (16 and 14 pt) fonts, respectively, in bold face shall be flushed to the left (not centered) with 12 point and 6 point spacing respectively above and below these lines. In further subdivisions character size of 3 and 3.5 with bold face, small caps, all caps and italics may be used for the titles flushed left or centered. These shall not feature in the contents.

\subsection{Table / Figure Format}

As far as possible tables and figures should be presented in portrait style. Small size table and figures (less than half of writing area of a page) should be incorporated within the text, while larger ones may be presented on separate pages. Table and figures shall be numbered chapter wise.
\par For example, the fourth figure in chapter 5 will bear the number Figure 5.4 or Fig 5.4. Table number and title will be placed above the table while the figure number and caption will be located below the figure. Reference to Table and Figures reproduced from elsewhere shall be cited at the end. A separate line in the table and figure caption, e.g. (after McGregor[12]).
\subsection{Non-paper Material}
Digital or magnetic materials, such as CDs and DVDs may be included in the thesis. They have to be given in a closed pocket on the back cover page of the thesis. It should be borne in mind that their formats may become obsolete due to rapid change in technology, making it impossible for the Central Library to guarantee their preservation and use.
\par All the non-paper materials must have a label each indicating the name of the thesis, name of the student and the date of submission.

